%---------------------------------------------------------------------------
% Preface
% asterisk signals a title without numbering
%\chapter*{Preface}
%\

%Blah blah \dots

% \clearpage
%---------------------------------------------------------------------------
% Abstract

%\chapter*{Zusammenfassung}
% \addcontentsline{toc}{chapter}{Zusammenfassung}

%Bla bla \dots

\cleardoublepage

\chapter*{Abstract}
 \addcontentsline{toc}{chapter}{Abstract}
 XXXstart with a questions maybeXXX
 Can social media data function as a supplement to traditional survey data to evaluate spatial distributions of nature-based recreation activities?

Blah blah \dots

\cleardoublepage

\chapter*{Acknowledgements} \label{acknowledgements}
  \addcontentsline{toc}{chapter}{Acknowledgements}
I would like to express my very great appreciation to all my supervisors, friends and institutions that helped me during the course of this thesis. First of all I want to thank ETH for providing me with the infrastructure, research literature and needed resources that allowed the start and completion of this thesis.\\
Personally I want to sincerely thank Dr. Marcel Hunziker for approving my master thesis proposal that originated from a term paper of the 'Advanced Landscape Research' lecture which he also supervised. \\
I would like to offer my special thanks to Dr. Flurina Wartmann from the WSL who was my primary contact throughout the entire thesis and who assisted me in setting up the ground truthing as well as in various administrative concerns. \\
Prof. Dr. Ross Purves and Dr. Rahul Deb Das from the geocomputation unit of the University of Zurich provided me with expert knowledge in all programming related issues for which I am deeply grateful.\\
\newline
Additionally, I would like to thank the spatial planing department of the Canton of Zug - especially Martina Brennecke and Alexander Gnos. Former for her personal assistance and insight on relevant projects in the research area and latter for his provision of needed geo-data in the form of shapefiles. My special thanks are extended to the staff of the company Zugerbergbahn, particularly to Christoph Sidler for providing me with passenger numbers of their mountain railway. \\
\newline
I cannot thank my sister Florence enough for conducting the entire 'passive observation' - an essential part of the ground truthing. This part would not have been possible without her. And finally I want to thank and mention my friends Achilleas Lehmann and Christopher O'Bryan who proofread my thesis and gave constructive feedback that lead to this final version of my thesis.

\cleardoublepage
%---------------------------------------------------------------------------
% Table of contents

 \setcounter{tocdepth}{2}
 \tableofcontents
 \listoftables\addcontentsline{toc}{chapter}{List of Tables}
 \listoffigures\addcontentsline{toc}{chapter}{List of Figures}

 \clearpage

%---------------------------------------------------------------------------
% Symbols

\chapter*{Nomenclature}\label{chap:symbole}
 \addcontentsline{toc}{chapter}{Nomenclature}

%\section*{Symbols}
%\begin{tabbing}
% \hspace*{1.6cm} \= \hspace*{8cm} \= \kill
% $\mathrm{EHC}$ \> Conditional equation \> [$-$] \\[0.5ex]
% $e$ \> Willans coefficient \> [$-$] \\[0.5ex]
% $F,G$ \> Parts of the system equation \> [\unitfrac[]{K}{s}]
%\end{tabbing}

%\section*{Indicies}
%\begin{tabbing}
% \hspace*{1.6cm}  \= \kill
% a \> Ambient \\[0.5ex]
% air \> Air
%\end{tabbing}

\section*{Acronyms and Abbreviations}
\begin{tabbing}
 \hspace*{1.6cm}  \= \kill
 API \> Application Program Interface \\
 CSV \> Comma Separated Values \\
 ETH \> Eidgen\"{o}ssische Technische Hochschule \\
 FOEN \> Federal Office for the Environment \\
 FOPH \> Federal Office for Public Health \\
 GIS \> Geographical Information System \\
 GPS \> Global Positioning System \\
 GUI \> Graphical User Interface \\
 HTML \> Hypertext Markup Language \\
 IR \> Information retrieval \\
 JSON \> JavaScript Object Notation \\
 M1 \> Machine learning model NR.1 that was trained on text-data \\
 M2 \> Machine learning model NR.2 that was trained on text-data and image-data \\
 ML \> Machine Learning \\
 NBRA \> Nature-based Recreation Activity \\%[0.5ex]
 NLTK \> Natural Language processing Toolkit \\ 
 RDBMS \> Relational Database Management Systems \\
 SMD \> Social Media Data \\
 SMP \> Social Media Platform \\
 SVM \> Support Vector Machine \\
 SVC \> Support Vector Classifier \\
 SQL \> Structured Query Language \\
 TF-IDF \> Term Frequency over Inverse Document Frequency \\
 TOS \> Terms of Service \\
 URL \> Uniform Resource Locator \\ 
 VGI \> Volunteered Geographic Information
\end{tabbing}

\section*{Definitions} \label{definitions}
% hspace*{1.6cm}  \= \kill
\texttt{API:} Application interfaces allow the controlled access of programs to specific services of a software system. 
\newline

\texttt{API endpoint:} Describes a specific service or communication channel a user can access to interact with an API. Different API endpoints provide the user with different data.
\newline

\texttt{Bulk upload:} In the context of this thesis, bulk uploads describe a certain amount of simultaneously or during a short period of time uploaded media objects to the same social media platform by a single author.
\newline

\texttt{Feature/word-token/token:} All these terms describe the features of a text-classification model and have identical meaning in the context of this thesis.
\newline

\texttt{Media object:} All user generated social media data that has been acquired through a API or third party company will be referred to as media objects. In the sense of machine learning the term media object can be used interchangeably with the term 'document'.
\newline

\texttt{Netlytic:} A third party company which provides licensed data of numerous social media platforms (among others Instagram) based on user specific queries.
\newline

\texttt{Pipelining:} Sequence of functions where the output of the preceding function is used as input for the successive function.

\texttt{Regular expression:} A sequence of characters that are used in programming to define string patterns are referred to as regular expressions or also 'regex'.

\cleardoublepage

%---------------------------------------------------------------------------
