\chapter{Ethics}

Drawing upon crowed sourced big-data comes inevitably with ethical questions. Using GPS locations and other sensitive data to identify and analyse patterns of human behaviour is rightfully criticised and sometimes considered unethical practise even by scientists <source>.
We are only talking about data that is publicly available, data that the author decided to share on the web on his or her own regard. But did th author sign up for his or her data to be systematically analysed? This depends on the AGB (change term) of the corresponding SMP.
One has to understand the potential power this data possess. Having access to hundreds of personalised media objects of a specific social media user enables analytics to construct a blueprint of that person. This blueprint can encompasses the persons political orientation, its  home and work location, preferences and dislikes as well as daily routines. Personalised advertisement would then be the smallest threat that could come from it <source>.
With that out of the way, I personally think that the ethical discrepancy is legitimately continuously discussed in this context. Showing how the data and the anonymity of the authors is treated as well as the goal a project pesues are key elements that determine in my opinion if the ethicality is violated or not. Projects with commercial intentions cross that line according to many social scientists and data privacy analysts <source>. 

WHAT IS THE GOAL OF THE DATA – COMMERCIAL? SCIENCE?
HOW WELL IS THE ANONYMITY OF THE AUTHOR PROTECTED?
IS THE IDENTITY OF THE AUTHOR OF INTEREST FOR THE ANALYSIS?

- ethics - Picture with military equipment and their exact location. !!!
(https://www.flickr.com/photos/danielebneter/1412718248/)
(https://www.flickr.com/photos/23649191@N02/10458196724)


WRITE ABOUT THE MISSLES FOUND ON A PICTURE → CRITICAL MILITARY INFORATION!

not everything entirely of social media data is volunteerely shared:
1.See, L.; Mooney, P.; Foody, G.; Bastin, L.; Comber, A.; Estima, J.; Fritz, S.; Kerle, N.; Jiang, B.; Laakso, M.; et al. Crowdsourcing, Citizen Science or Volunteered Geographic Information? The Current State of Crowdsourced Geographic Information. ISPRS Int. J. Geo-Inf. 2016, 5, 55. [CrossRef]

Kanakaris, Venetis and Tzovelekis, Konstantinos and Bandekas, D.V. 2018
Geo-Location Twitter And Instagram Based On OSINT Techniques: A Case Study

