\chapter{Ethics}\addcontentsline{toc}{chapter}{Ethics}

Drawing upon crowed sourced big-data is inevitably shadowed by ethical controversies. Using GPS locations and other sensitive data to identify and analyse patterns of human behaviour is rightfully criticised and sometimes considered unethical practise.\\
We are only talking about data that is publicly available, data that the user decided to share on the web on his or her own behalf - or is it?. The paper of \parencite{Estima2016} shows that not the entirety of data is voluntarily shared. Did the user sign up for his or her data to be systematically analysed? This depends on the Terms of Service (TOS) of the corresponding SMP which are hardly ever read by the consumer.\\
One has to understand the potential power this data possess. Having access to hundreds of personalised media objects of a specific social media user enables analytics to construct a blueprint of that person. This blueprint can encompasses the persons political orientation, its home and work location, preferences and dislikes as well as daily routines. Personalised advertisement would then be the smallest resulting threat. Interpersonal surveillance \parencite{Trottier2017}, stalking \parencite{Lyndon2011} or potential blackmailing through acquired sensitive information are by far greater and current threats. Additionally, it is unclear from a data-researcher point of view how to deal with encounters of media objects that indicate illegal behaviour e.g littering, assault or theft.\\

With that being said I personally think that the ethical discrepancy is legitimately continuously discussed in this context. Showing how the data and the anonymity of the authors is treated as well as the goal a project pursues are key elements that determine in my opinion if ethical terms are violated or not. 





