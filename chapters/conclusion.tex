\chapter{Conclusions} \label{conclusion_outlook}

Two ML models named M1 and M2 were trained on Instagram media objects originating from the Canton of Zurich to predict NBRAs contained in Instagram and Flickr media objects from the Canton of Zug. Both models were tuned on the linearSVC and randomForest fitting algorithms of which the former performed best. The creation and comparison of two models served the purpose to investigate the effect of combining text and image contained information on model performance compared to only using text-data for model training.\\
M2 which is trained on said data-combination proofed to perform better on new (unseen) data than M1. Not only did M2 have a slightly better overall precision but it also identified twice as many media objects that contained NBRAs. Therefore, by reducing the False Negative (FN) predictions M2 achieved to increase its recall compared to its predecessor M1. A comparison to recently created baseline-models of \parencite{Das2018, Li2018} revealed similar test accuracies of but rather at the lower end. These baseline-accuracies ranged from 89.91\% to 96.23\% whereas M1 and M2 achieved a test accuracy of 93.7\% and 88.2\% respectively (see figure \ref{tab:m1_linearSVC_bestscores} and figure \ref{tab:m2_linearSVC_bestscores}).\\
The visual matching of M2 predicted NBRA-occurrences to the ground truth data revealed plausible overlaps. This outcome supports the usage of SMD for similar tasks which could be an addition to or partly replace costly traditional methods. The ground truth additionally revealed interesting insights on the social media usage of 52 interviewees. Instagram and Facebook were as expected the dominant SMPs. Flickr on the other hand was not mentioned once (see figure \ref{fig:interview_SMP}). The age distribution of social media users was estimated to be of similar proportions  to sources such as \parencite{2013} with the same dominant age group of 21-30.

\section*{Future prospects:} The presented models in this thesis are far from their full potential. Small training's datasets as well as further expendable data-processing show opportunities for method refinement. Instagram location accuracy could potentially be proved by utilising topographic keywords in the user-generated text to narrow down the media object origin similarly to \parencite{Ostermann2015}. \\
Additionally, different ML meta-algorithms should be investigated. One of which would be \textit{Ensemble Learning}. This approach combines multiple models with potentially different fitting algorithms to a single model \parencite{Zhou2009}. Another option would be \textit{Stacking} which is similar to \textit{Ensemble Learning} in the sense that multiple models are trained. Auxiliary a final model is trained based on the predictions of the individual models on the original training's data. This approach has been shown to be extremely powerful and is currently used by Netflix to provide users with suitable movie suggestions \parencite{AndreasToscher2009}. The advantages of both approaches are increased prediction accuracies, reduced bias and less overfitting compared to individual models. 



future improvements: better trainingdata (area consistent, size, mix SMP), improved data-processing. improved location setting by utilising used topographic terms
conduct Statistical similarity analysis
Add: hierarchical (stacking ensemble classifier; blender)!!!!

    Conclusions of the main findings summarised
    This section should only contain information, which has been discussed earlier, no new information should be introduced
    Conclusions generally do not contain literature references
    It is possible to lay also an outlook based on the findings (new research questions and future directions, which would bring the topic further)


