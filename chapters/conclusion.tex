\chapter{Conclusions} \label{conclusion_outlook}

In this thesis, two ML models named M1 and M2 were trained on Instagram media objects originating from the Canton of Zurich to predict NBRAs contained in Instagram and Flickr media objects from the Canton of Zug. Both models were tuned on the linearSVC and Random Forest fitting algorithms of which the former performed best. The creation and comparison of two models served the purpose to investigate the effect of combining text and image contained information on model performance compared to only using text data for model training.\\
M2 which is trained on said data-combination proofed to perform better on new (unseen) data than M1. Not only did M2 have a slightly better overall precision but it also identified twice as many media objects that contained recreational activities. Therefore, by reducing the False Negative (FN) predictions or Type II error M2 achieved to increase its recall compared to its predecessor M1. A comparison to recently created baseline-models of \parencite{Das2018, Li2018} revealed similar test accuracies of but rather at the lower end. These baseline-accuracies ranged from 89.91 \% to 96.23 \% whereas M1 and M2 achieved a test accuracy of 93.7 \% and 88.2 \% respectively (see figure \ref{tab:m1_linearSVC_bestscores} and figure \ref{tab:m2_linearSVC_bestscores}). \\

Both models showed below average class-specific performances for the recreation activities 'jogging' and 'picnic'. This issue is among others caused by word-tokens with multiple use-cases such as the action verb \textit{to run} of the class 'jogging' as well as difficulties of the models to capture the entirety of components that make
out a class. Increasing the available training data used for model creation is considered to allow significant further performance improvements.\\ 

The visual matching of M2 predicted recreational activity occurrences to the ground truth data revealed plausible overlaps. This outcome supports the usage of social media data for similar tasks which could be an addition to or partly replace costly traditional methods. The ground truth additionally revealed interesting insights on the social media usage of 52 interviewees. Instagram and Facebook were as expected the dominant social media platforms. Flickr on the other hand was not mentioned once (see figure \ref{fig:interview_SMP}). The age distribution of social media users was estimated to be of similar proportions to sources such as \parencite{2013} with the same dominant age group of 21-30.

\section{Main findings}

\begin{itemize}
  \item Training a machine learning model on both processed text data and labels of structural image elements yields a performance increase compared to exclusively text trained models by significantly reducing the Type II error and enhancing the overall model precision.
  \item VGI in the form of georeferenced social media data from Instagram and Flickr show proxy potential to predict spatial occurrences of recreational activities.
  \item 'Hiking' and 'walking' are by far the most dominant recreation activities in the research area according to both ground truth and social media data.
  \item \textit{Geographic closeness} followed by the presence of water bodies are the most decisive reason for choosing a location for recreation according to the performed interviews.
  \item The social media platforms Instagram and Facebook hold the highest popularity among the interviewees.
  \item Social media active people are on average younger than people that do not engage in social media.
  \item The linear support vector machine classifier performs better than the decision tree based Random Forest fitting algorithm for this text-classification tasks. 
\end{itemize}

\section{Future prospects} \label{future_prospects}
The presented models in this thesis are far from their full potential. Small training datasets from inconsistent sources as well as further expendable data-processing show opportunities for method refinement. Instagram location accuracy could potentially be improved by utilising topographic keywords in the user-generated text to narrow down the media object origin similarly to \parencite{Ostermann2015}. Text embedded emojis could be linked to plausible recreational activities to further improve model performance.\\
Additionally, different ML meta-algorithms should be investigated. One of which would be \textit{Ensemble Learning}. This approach combines multiple models with potentially different fitting algorithms to a single model \parencite{Zhou2009}. Another option would be \textit{Stacking} which is similar to \textit{Ensemble Learning} in the sense that multiple models are trained. Auxiliary a final model is trained based on the predictions of the individual models on the original training data. This approach has been shown to be extremely powerful and is currently used by Netflix to provide users with suitable movie suggestions \parencite{AndreasToscher2009}. The advantages of both approaches are increased prediction accuracies, reduced bias and less overfitting compared to individual models. Furthermore, a more sophisticated VGI proxy potential could be estimated by conducting a statistical similarity analysis between the passive observation and the social media data inferred occurrences of recreational activities by using either Ripley's K Function or Moran's \textit{I} statistics in different radiuses around the three ground truth locations \parencite{OSullivan2003}. Similar methodology could be applied to Foursquare infrastructural data to further assess the correspondence to the model inferred signals while filtering the venues to obtain more recreational activity specific subgroups to enhance the spatial comparison. 



