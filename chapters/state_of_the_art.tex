\chapter{State of the Art} \label{state_of_the_art}
This chapter will cover related scientific work regarding the usage and application of UGC, comparisons of different data-sources, nature-based recreation as well as text-classification models. The presented key literature starts out at providing a general overview of the mentioned topics and becomes more specific in the progress. According to the presented current state of the art the research gap this thesis tries to fill is displayed and put into context.

\section{Applications of UGC}
The following subsections provides a general overview of the wide scientific spectrum for which UGC or more specifically VGI and SMD have been used.

\subsection{Quantifying ecosystem services}
Ecosystem services (ES) describe nature given benefits which are provided by functioning environments and their incorporated ecosystems. Due to the complexity and intangibility of ES their associated (monetary) value is hard to estimate. The value of \textit{provisioning} ES for instance such as 'clean water provision' can be estimated by calculating the water filtration cost by state-of-the-art filter machines. The potential value of \textit{cultural} ES such as the aesthetic value of landscapes is harder to grasp since it depends on the individual benefit people associate with it. Through UGC - in particular in the form of SMD - people express their personal perception which has been used for the subsequent papers to estimate values of specific ES also as alternative to conventional methods such as interviews and surveys.  
\paragraph*{\textcite{Richards2018}} applied similarly to this thesis the \textit{Google Cloud Vision} image recognition algorithm to 20'000 retrieved geo-tagged Flickr photographs of Singapore. The paper proposed the use of said photographs for the mapping of the cultural use or appreciation of ecosystems by automating the image content analysis process. Grouping by hierarchical clustering revealed that more than 20\% of the images were taken of nature, more precisely of animals and plants which were located dominantly around specific natural attractions, parks or generally areas with high vegetation cover. The approach developed by the paper is accurate and saved an estimated 170 hours of human image content analysis. As summary, the method provided by \textcite{Richards2018} allows the scalable mapping and assessment of cultural ecosystems services which can be easily applied to different areas of interest.

\paragraph*{\textcite{Figueroa-Alfaro2017}} used Panoramio\footnote{https://www.panoramio.com/, out of service since 04.11.2016} and Flickr geo-tagged photographs as a type of crowed-sourced alternative to traditional methods to examine the relationships between aesthetic value and citizens' activities in Nebraska. The outcome of the spatial clustering of the photographs with the cluster/ outlier analysis tool from ArcGIS\footnote{https://www.arcgis.com/index.html, accessed: 01.04.2019} identify existing and new areas with aesthetic value. Anselin Local Moran's I statistic was used to find points that were statistically significant at a 95\% confidence level. \citeauthor{Figueroa-Alfaro2017} underlined the importance of the findings also for urban planning and environmental management. Knowing which areas people prefer to visit helps to allocate available resources more efficiently and supports the preservation as well as conservation of natural resources from which simultaneously the local tourism profits. \\
\newline
A similar study was performed by \textcite{Yoshimura2017} that made use of viewsheds of geo-tagged Flickr photographs of Hokkaido, Japan. \citeauthor{Yoshimura2017} proposed that areas with many overlapping viewsheds represent areas of high aesthetic and therefore cultural value since they were the focus of many visitors. Detecting the photographic orientation as already proclaimed by \textcite{Unknown2013} was problematic. Also the overestimation of popular and highly visited sites was recorded as potential bias which is reflected in the resulting areas of interest mapping.


\subsection{Visitor rates \& monitoring}
Good management and marketing relies heavily on customer information. The same applies to the environmental management of recreation sites, protected or conservation areas. Detailed knowledge about the visitors from which the projects live is key. The following papers will present data-acquisition approaches to answer questions related to visitor behaviour.  

\paragraph*{\textcite{Heikinheimo2017}} used the most popular national park named Pallas-Yll\"astunturi in Finland as research area to investigated visitor behaviour. This encompassed the identification of sub-areas in the park with the highest visitation rates, spatial and temporal patterns of performed activities, reasons for visit and information about the demography as well as home location of visitors. The data to answer those questions was collected through 1'927 visitor surveys and 19'939 geo-tagged Instagram images from 7'700 unique users. The aim of this two sided approach was to test to what extent social media content reflects or complements the reported experiences extracted from the traditional methods. This controversy is again debated in section \ref{data_comparison_SotA}. The data-comparison and research questions of \textcite{Heikinheimo2017} are comparable to this thesis but the applied methods differ.
The highly visited areas were identified according to the spatial density distribution of the geo-tagged Instagram images. These images were manually classified according to the performed activities where displayed equipment played an important role. Potential home location of 291 Instagram users that visited the park were detected by analysing the country or region from which the most photos were uploaded. \\
The results revealed that social media can provide a source for continuous monitoring as a rapid and cost-efficient alternative to traditional surveys. With the help of the social media data it was possible to detect the most popular sub-regions in the park. Also, the Instagram data revealed performed activities that were not captured by the survey. Walking and hiking had the highest recorded occurrence according to both data-sources which corresponds to the findings of this thesis. Lastly, \textcite{Heikinheimo2017} proclaims potential for "using Instagram photographs for group-size estimation for small to medium sized groups (2-10 people), but not necessarily for detecting people travelling alone or in very large groups." 

\paragraph*{\textcite{Pettersson2011}} examined the usage of GPS-data as 'crowdsensing' data-foundation to understand the time and space movement of visitors to the Biathlon World Championships 2008 in \"Ostersund, Sweden. In addition to equipping people with GPS-devices which additionally enabled them to input personal experiences to a certain location, questionnaires were used to study the visitors movements and experiences also. The aim of the study was the evaluate the practicability of GPS-data during an outdoor sport event. \\
The results of the paper allowed for a comparison of the strengths and weaknesses of the UGC and the questionnaire data which suggested a combination of both methods to yield the best information retrieval. In conclusion, the usage of crowdsourced GPS-data was successful to grasp the macro-movements and experiences of the visitors. This paper provides a good transition to research that tries to quantify bigger-scale movements of people in space.

\subsection{Mobility patterns}
The mobility of people is an highly valued necessity. Correspondingly, the planing and optimisation of transportation systems such as road nets especially in urban areas is important. 


\paragraph*{\textcite{Barchiesi2015}} derived mobility-patterns for the UK using a machine learning algorithm trained on 8 million geo-tagged images of 16'000 individual Flickr users between 2007 and 2013. The model is capable of inferring the probability of people being in a certain location as well as the probability of trajectory-movement between different locations. Considering Flickr-data as a proxy for human mobility comes with the limitation that pictures from the same user with long time-spawns between them potentially lead to false or incomplete travel-pathways. \textcite{Barchiesi2015} were able to show that this issue is mitigated by the fact "that the time elapsed between consecutive photos follows a heavy tailed distribution [\dots] with most photos taken only a few hours or a few days apart, and only a small number of photos taken as much as a few years apart." \\
The paper concludes that there is fundamental evidence that Flickr derived data can be used to quantify human mobility. Due to a lack of official survey data the presented model could only be partially validated but they appear to be in general agreement.

\paragraph*{\textcite{Grossenbacher2014}} conducted an commuter mobility-analysis for entire Switzerland. 1 year worth of geo-tagged Twitter\footnote{https://twitter.com/?lang=en, accessed: 01.04.2019} posts were used as data-foundation to extract transportation flows between major Cities. An intensive literature review revealed flaws of preceding work. \citeauthor{Grossenbacher2014} points out that often times mobility-patterns were inferred from "mere spatial aggregations of individually unreferenced, decoupled content." Correspondingly, the thesis presented an approach of linking Twitter users to specific municipalities inside of Switzerland while estimating home and work locations. This in conjunction with a sophisticated pre-processing of the data enabled the comparison of the results to official data.\\
The conclusion states that Twitter-data hints at a "socially, culturally, and economically unequally represented population, which raises concerns about the validity of inferences made with such data." Nevertheless it can be used as a potential data-source for studying human-mobility since it provides good indicators for actual flows of people in regions where sufficient geo-tagged data is available.

\subsection{Disease outbreak tracking}
Infectious disease epidemiology has recognised big data such as UGC as novel potential sources to complement traditional infectious disease surveillance. The real-time data-acquisition with included locational information from multiple data-streams allows for more accurate predictions and could help formulate future disease-mitigation strategies. Additionally, problems regarding determining origins of epidemic outbreaks as well as identifying locations with risk of new outbreaks will among others be addressed in the subsequent papers.

\paragraph*{\textcite{Lee2016}}

\paragraph*{\textcite{Schmidt2012}}




\section{Data-comparison} \label{data_comparison_SotA}
\subsection{SMD from different SMP} \label{SMP_vs_SMP}
Tenkanen2017

\subsection{UGC to empirical data} \label{ugc_vs_empirical}
The previous section presented the application-broadness of UGC. The question arise how well does this data and the therefrom drawn conclusion represent reality? Empirical data is considered to be the closest approximation to reality. This section will introduce papers which investigate the legitimacy of UGS (including SMD) as proxy for real world conditions compared to empirical data acquired through conventional methods such as interviews and surveys.

- (Wood et al. 2013) - nur Flickr, globale Untersuchung, Vergleich von Flickr Posts und eigentlichen Besucherfrequenzen
 Resultat: Crowed-sourced information korreliert gut mit empirischer Information wo Leute hingehen

- (Sonter et al. 2016) - nur Flickr, Natur-basierte Naherholungsuntersuchung im Naturschutzgebiet Vermont USA, Vergleich von Flickr und eigentlichen Besucherfrequenzen und Berechnung der durchsch. Reisekosten und somit ‘Wert’ des Gebietes.
Resultat: ebenfalls sign. Korrelation zwischen Flickr und Umfragebesucherraten. Erholung macht mehr Wert aus als Holzschlag, Holzprodukte, Holzenergie, Papierindustrie zusammen.

xxxmaybe also input some graphs from the papersxxxx

- Wartmann2018

\section{Nature-based recreation assessments}
xxxxINSERT SHORT INTROxxxx
\subsection{with empirical data from conventional methods}
Weyland2014 - Outdoor recreation, as a cultural ecosystem service. Ecosystem Services framework is used to estimate the ecosystem service supply and benefit of different landscapes based on nine landscape metrics such as land cover, land use, at a coarse and campsite scale for the country Argentina. The examined landscapes showed difference in their capacity of outdoor recreation potential. Crop area mostly accounted for positive correlations on the capacity of outdoor recreation potential contrary to forest cover. Also road and population density were proven to show for higher benefits.

Sen2014 - The goal of the paper was to present a general tool for recreation planning and environmental decision making since 2'858 million outdoor recreational visits were made in England during 2010, entailing direct expenditure of over £20 billion. Therefore, a spatially sensitive prediction of outdoor recreation visits and values for different ecosystems in the United Kingdom was conducted. Their data-foundation consisted of outset and destination characteristics and locations combined with survey information from over 40'000 households. This data was used to create a trip generation function (TGF) to predict visitation numbers and their associated values.\\
The result provided by the GIS powered tool showed the importance of travel and time costs, site characteristics, substitute availability and socioeconomic and demographic factors in determining the pattern and level of recreational demand.

Neuvonen2010 - This paper investigates the relationship between the number of visits to 35 national parks in Finland and their characteristics via linear regression modelling. The understanding of these relationships is crucial for park planing and management. Considered were the natural characteristics, the recreation facilities and services inside a park and tourist services in surrounding communities. \\
The results of the study revealed that the variety of recreation opportunities, the number of biotopes, the provision of trails and the park's age increase the number of visits. It was also shown that parks located close to populated areas dominantly fulfil recreation-provision services over simply being resource-based destinations.

\subsection{using UGC}

Monkman2018 - fishery (wildlife recreation activity)

Mancini2018
    
\section{Existing text-classification models}
    - REGARDING TEXT-CLASSIFICATION MODELS
    
Li2018 - 
Das2018 - 





    - Conventional methods for information retrieval 
    - what has been done with SMD in general
    - what text classification models have been trained on SMD?
    - Explicitly applied models for SMD    







\section{Research gap}