\chapter{State of the Art} \label{state_of_the_art}
This chapter will cover related scientific work regarding the usage and application of UGC, comparisons of different data-sources, nature-based recreation as well as text-classification models. The presented key literature starts out at providing a general overview of the mentioned topics and becomes more specific in the progress. According to the presented current state of the art the research gap this thesis tries to fill is displayed and put into context.

\section{Applications of UGC-data}
The following subsections provides a general overview of the wide scientific spectrum for which UGC or more specifically VGI and SMD have been used.

\subsection{Quantifying ecosystem services}
Ecosystem services (ES) describe nature given benefits which are provided by functioning environments and their incorporated ecosystems. Due to the complexity and intangibility of ES their associated (monetary) value is hard to estimate. The value of \textit{provisioning} ES for instance such as 'clean water provision' can be estimated by calculating the water filtration cost by state-of-the-art filter machines. The potential value of \textit{cultural} ES such as the aesthetic value of landscapes is harder to grasp since it depends on the individual benefit people associate with it. Through UGC - in particular in the form of SMD - people express their personal perception which has been used for the subsequent papers to estimate values of specific ES also as alternative to conventional methods such as interviews and surveys.  
\paragraph*{\textcite{Richards2018}} applied similarly to this thesis the \textit{Google Cloud Vision} image recognition algorithm to 20'000 retrieved geo-tagged Flickr photographs of Singapore. The paper proposed the use of said photographs for the mapping of the cultural use or appreciation of ecosystems by automating the image content analysis process. Grouping by hierarchical clustering revealed that more than 20\% of the images were taken of nature, more precisely of animals and plants which were located dominantly around specific natural attractions, parks or generally areas with high vegetation cover. The approach developed by the paper is accurate and saved an estimated 170 hours of human image content analysis. As summary, the method provided by \textcite{Richards2018} allows the scalable mapping and assessment of cultural ecosystems services which can be easily applied to different areas of interest.

\paragraph*{\textcite{Figueroa-Alfaro2017}} used Panoramio\footnote{https://www.panoramio.com/, out of service since 04.11.2016} and Flickr geo-tagged photographs as a type of crowed-sourced alternative to traditional methods to examine the relationships between aesthetic value and citizens' activities in Nebraska. The outcome of the spatial clustering of the photographs with the cluster/ outlier analysis tool from ArcGIS\footnote{https://www.arcgis.com/index.html, accessed: 01.04.2019} identify existing and new areas with aesthetic value. Anselin Local Moran's I statistic was used to find points that were statistically significant at a 95\% confidence level. \citeauthor{Figueroa-Alfaro2017} underlined the importance of the findings also for urban planning and environmental management. Knowing which areas people prefer to visit helps to allocate available resources more efficiently and supports the preservation as well as conservation of natural resources from which simultaneously the local tourism profits.

\paragraph*{\textcite{Yoshimura2017}} 

Oteros-Rozas2016 -

\subsection{Visitor rates \& monitoring }
Heikinheimo2017 - monitoring

Pettersson2011

\subsection{Mobility patterns}
Barchiesi2015 - 

Grossenbacher2014

\subsection{Disease outbreak tracking}
Lee2016

Schmidt2012



\section{Data-comparison}
\subsection{SMD from different SMP}
Tenkanen2017

\subsection{UGC to empirical data}
The previous section presented the application-broadness of UGC. The question arise how well does this data and the therefrom drawn conclusion represent reality? Empirical data is considered to be the closest approximation to reality. This section will introduce papers which investigate the legitimacy of UGS (including SMD) as proxy for real world conditions compared to empirical data acquired through conventional methods such as interviews and surveys.

- (Wood et al. 2013) - nur Flickr, globale Untersuchung, Vergleich von Flickr Posts und eigentlichen Besucherfrequenzen
 Resultat: Crowed-sourced information korreliert gut mit empirischer Information wo Leute hingehen

- (Sonter et al. 2016) - nur Flickr, Natur-basierte Naherholungsuntersuchung im Naturschutzgebiet Vermont USA, Vergleich von Flickr und eigentlichen Besucherfrequenzen und Berechnung der durchsch. Reisekosten und somit ‘Wert’ des Gebietes.
Resultat: ebenfalls sign. Korrelation zwischen Flickr und Umfragebesucherraten. Erholung macht mehr Wert aus als Holzschlag, Holzprodukte, Holzenergie, Papierindustrie zusammen.

xxxmaybe also input some graphs from the papersxxxx

- Wartmann2018

\section{Nature-based recreation assessments}
xxxxINSERT SHORT INTROxxxx
\subsection{with empirical data from conventional methods}
Weyland2014 - Outdoor recreation, as a cultural ecosystem service. Ecosystem Services framework is used to estimate the ecosystem service supply and benefit of different landscapes based on nine landscape metrics such as land cover, land use, at a coarse and campsite scale for the country Argentina. The examined landscapes showed difference in their capacity of outdoor recreation potential. Crop area mostly accounted for positive correlations on the capacity of outdoor recreation potential contrary to forest cover. Also road and population density were proven to show for higher benefits.

Sen2014 - The goal of the paper was to present a general tool for recreation planning and environmental decision making since 2'858 million outdoor recreational visits were made in England during 2010, entailing direct expenditure of over £20 billion. Therefore, a spatially sensitive prediction of outdoor recreation visits and values for different ecosystems in the United Kingdom was conducted. Their data-foundation consisted of outset and destination characteristics and locations combined with survey information from over 40'000 households. This data was used to create a trip generation function (TGF) to predict visitation numbers and their associated values.\\
The result provided by the GIS powered tool showed the importance of travel and time costs, site characteristics, substitute availability and socioeconomic and demographic factors in determining the pattern and level of recreational demand.

Neuvonen2010 - This paper investigates the relationship between the number of visits to 35 national parks in Finland and their characteristics via linear regression modelling. The understanding of these relationships is crucial for park planing and management. Considered were the natural characteristics, the recreation facilities and services inside a park and tourist services in surrounding communities. \\
The results of the study revealed that the variety of recreation opportunities, the number of biotopes, the provision of trails and the park's age increase the number of visits. It was also shown that parks located close to populated areas dominantly fulfil recreation-provision services over simply being resource-based destinations.

\subsection{using UGC}

Monkman2018 - fishery (wildlife recreation activity)

Mancini2018
    
\section{Existing text-classification models}
    - REGARDING TEXT-CLASSIFICATION MODELS
    
Li2018 - 
Das2018 - 





    - Conventional methods for information retrieval 
    - what has been done with SMD in general
    - what text classification models have been trained on SMD?
    - Explicitly applied models for SMD    







\section{Research gap}