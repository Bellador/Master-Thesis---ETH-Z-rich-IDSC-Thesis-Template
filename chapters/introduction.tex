\chapter{Introduction}
People are now sharing information more than ever on social media platforms. This information encompasses interests, performed activities and intentions. This social media derived big data holds a lot of potential for various kinds of research applications due to its comparably fast, continuous data acquisition as well as its good spatial and temporal coverage. <INSERT REFERENCE>. HERE REFERENCE SOME OTHER PAPERS THAT WORKED WITH SOCIAL MEDIA \\
\newline
Today, job induced mental fatigue, stress and illness are a side product of an ever more specialized economy. The Swiss government responded to this growing thread by openly addressing the issue and by motivating the cantonal administrations to foster recreation among the population <INSERT REFERENCE>. To achieve this, data must be acquired to take the appropriate steps to an improved provision of recreation.\\
This thesis tries to respond to that current need for information by investigating the potential of social media data (SMD) - in particular \textit{Instagram} and \textit{Flickr} - for analysing the spatial occurrences of nature-based recreation activities (NBRAs) including walking, jogging, hiking, dog walking, biking, horse riding and picnicking as well as their temporal fluctuations in the Canton of Zug, Switzerland. The results of this study aim to improve the efficient allocation of infrastructure in correspondence to the people’s usage of space. \\
\newline
This approach encompasses the processing of georeferenced social media posts which are referred to as media objects throughout this thesis. These \textit{Instagram} and \textit{Flickr} media objects generally consist of user generated text, an image, a location and an upload time. The idea is to extract information from the images via a deep-learning image recognition algorithm that detects structural elements and assigns appropriate text labels to them. The combination of these image derived text labels and the processed user generated text of 1\rq499 manually classified media objects is used to train a machine learning model to automate the content prediction of the mentioned NBRAs in any given media object. A comparison to conventional methods such as interviews and surveys will be drawn at the end to evaluate the models legitimacy. \\
The ethical discrepancy of using SMD will also be touched upon in an individual chapter.