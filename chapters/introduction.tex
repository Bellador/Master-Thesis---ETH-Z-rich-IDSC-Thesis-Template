\chapter{Introduction}
People are now sharing information more than ever on social media platforms which is also referred to as 'volunteered geographic information' (VGI). VGI encompasses among others people's interests, performed activities and future intentions \parencite{Goodchild2007}. This social media derived big-data holds a lot of potential for various research applications as already highlighted by papers such as \parencite{DiMinin2015}, \parencite{DiMinin2017} and \parencite{Meentemeyer2016}. The main advantages compare to conventional surveys are a comparably fast and (mostly) easy continuous data-acquisition as well as a good spatial and temporal coverage at multiple scales.
\newline
Today, job induced mental fatigue, stress and illness are side products of an ever more specialised economy. To address that issue the report of mental health in Switzerland \parencite{Ruesch2003} was conducted among others by the Federal Office for Public Health (FOPH) which highlighted the lack of prevention measures in place. This lead to a monitoring program to foster mental health provision and awareness in Switzerland \parencite{Schuler2012} also on cantonal level. The environment is know to be a sustainable source for mental health benefits by providing a diverse range of recreation possibilities as also noted in the Federal Office for the Environment (FOEN) report 'effects of the environment on human health' \parencite{Ragettli2017}.\\
\newline
Increasing the quality, quantity and attractiveness of nature-based recreation firstly requires knowledge over the existing usage of space in regards to activities people perform in the environment to answer the following questions: (1) What nature-based recreation activity (NBRA) should be promoted in a certain area and (2) what kind of supportive infrastructure shall there be constructed to maximise its utilisation? \\
This thesis tries to respond to that current need for information by investigating the potential of social media data (SMD) - in particular \textit{Instagram} and \textit{Flickr} data - for analysing the spatial occurrences of nature-based recreation activities (NBRAs) including walking, hiking, jogging, biking, dog walking, horse riding and picnicking in the Canton of Zug, Switzerland. The results of this study aim to improve spatial planing by efficiently allocating infrastructure in correspondence to the people's usage of space.\\
\newline
The approach presented in this thesis creates and evaluates two different machine learning (ML) models to predict NBRAs in georeferenced Instagram and Flickr media object based on a combination of text and image data. The mentioned social media platforms (SMPs) were proven to be good indicators for multiple applications such as describing: visitation rate to nature reserves \parencite{Heikinheimo2017, Keeler2015, Wood2013} or social events \parencite{Pettersson2011}, human mobility patterns \parencite{Barchiesi2015, Grossenbacher2014} and nature-based recreation locations \parencite{XY}. Different SMPs were also compared to each other by \parencite{Tenkanen2017}.  



This approach encompasses the processing of georeferenced social media posts which are referred to as media objects throughout this thesis. These \textit{Instagram} and \textit{Flickr} media objects generally consist of user generated text, an image, a location and an upload time. The idea is to extract information from the images via a deep-learning image recognition algorithm that detects structural elements and assigns appropriate text labels to them. The combination of these image derived text labels and the processed user generated text of 1'499 manually classified media objects is used to train a machine learning model to automate the content prediction of the mentioned NBRAs in a given media object. A comparison to conventional methods such as interviews and surveys will be drawn at the end to evaluate the models legitimacy. \\
The ethical discrepancy of using SMD will also be touched upon in an individual chapter.

- comparison traditional survey data with social media data


   -Scientific background and relevance of the subject
    -Focused review of the relevant literature
    -Research gaps, divergent points of view
    -Research question, aim(s) of the project, and precise hypothesis if applicable

Recommended tense: present (current knowledge), perfect (reference to research areas), past (reference to single studies).