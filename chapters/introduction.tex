\chapter{Introduction}
\section{Motivation}
Nowadays people publicly share a broad spectrum of information on social media platforms. This data is referred to as 'user generated content' or if it is georeferenced as 'volunteered geographic information' (VGI). VGI encompasses people's interests, performed activities, future intentions, sense-of-place and perception of locations \parencite{Goodchild2007}. This social media derived big-data holds a lot of potential for various research applications as already highlighted by papers such as \textcite{DiMinin2015, DiMinin2017, Meentemeyer2016}. The main advantages to conventional methods such as surveys and interviews are comparably fast, (mostly) easy and low-priced continuous data-streams as well as a good spatial and temporal coverage at multiple scales.\\
Also a phenomenon of modern times are increased rates of job induced mental fatigue, stress and illness which are side products of an ever more specialised and demanding economy. To address that issue the report of mental health in Switzerland \parencite{Ruesch2003} was conducted among others by the Federal Office for Public Health (FOPH) which highlighted the lack of prevention measures in place. This led to a monitoring program to foster mental health provision and awareness in Switzerland \parencite{Schuler2012} also on a cantonal level. The environment is known to be a sustainable source for mental health benefits by providing a diverse range of recreation possibilities. The extent and potential value associated with these nature-provided services was noted in the 'effects of the environment on human health' report \parencite{Ragettli2017} of the Federal Office for the Environment (FOEN).\\

\section{Problem statement and aim}
Increasing the quality, quantity and attractiveness of nature-based recreation primarily requires knowledge over the existing usage of space. More specifically in regards to the activities people perform in the environment to answer the following questions: (1) What nature-based recreation activity (NBRA) should be promoted in a certain area and (2) what kind of supportive infrastructure should be constructed to maximise its utilisation? This thesis tries to respond to that current need for information by investigating the potential of social media data (SMD) - in particular Instagram and Flickr data - for analysing the spatial occurrences of NBRAs including walking, hiking, jogging, biking, dog walking, horse riding and picnicking in the Canton of Zug, Switzerland. The mentioned social media platforms (SMPs) were proven to be good indicators for multiple applications such as determining visitation rates to nature reserves \parencite{Tenkanen2017, Heikinheimo2017, Keeler2015, Wood2013} or to social events \parencite{Pettersson2011}, human mobility patterns \parencite{Barchiesi2015, Grossenbacher2014} and recreation locations \parencite{Weyland2014, Hill2006, Neuvonen2010}. 

XXXXmore detail here how infrustructre information is obtained, what 

The SMP Foursquare was used as an indicator for the current dispersion of sport and recreation related infrastructure in the research area which was at the end compared to the identified spatial occurrences of NBRA.\\
The results of this study aim to improve spatial planing by efficiently allocating recreation infrastructure in correspondence to the people's usage of space. Additionally, the potential of SMD as a proxy for predicting peoples preferences as an alternative to conventional data-acquisition methods such as surveys is investigated.\\

\section{Approach}
The approach presented in this thesis creates and evaluates two machine learning (ML) models to predict NBRAs in georeferenced Instagram and Flickr posts. These posts are referred to as media objects throughout the thesis. The first model is trained on a combination of text and image data which has not in this way been done before to the assessment of the author. The second model was only trained on text data as a baseline to investigate the effect of including image content information. Structural image elements are extracted with the help of a deep-learning algorithm named Google Cloud Vision as also used by \textcite{Richards2018}. The models are individually trained on 1'046 manually classified Instagram media objects originating from a dataset of the region of Zurich \parencite{Gruzd2016} and tuned for best performance. Classification specific model evaluations are subsequently manually conducted on the NBRA-predictions in the research area located in the canton of Zug to compare the two finalised model performances based on untrained data. The innovation of this approach lies in the fully automated prediction process without any manual content analysis which uses dominantly openly accessible and free software. This shall pave the way for a feasible reconstruction and application by e.g. municipal authorities or agencies with a low budget for high priced software licences.
\newline
In the course of this thesis, additional ground truth data from interviews and passive observations was acquired in three locations of the research area. This information gives insight on the drivers which motivate people to visit the locations, what NBRAs they perform and their social media usage. The therefrom derived signals were compared to the SMD \parencite{Wartmann2018} to evaluate and justify the ML models legitimacy.


xxxBackgroundxxxx (Machine Learning)


xxxxEthicsxxxx
